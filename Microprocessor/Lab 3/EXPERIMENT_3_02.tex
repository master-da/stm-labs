\documentclass{article}
\usepackage[utf8]{inputenc}
\usepackage{graphicx}[]
\usepackage{url}
\usepackage{geometry}[margin=0.5in]
\usepackage{listings}
\usepackage{color}
\usepackage{float}
\usepackage{caption}
\usepackage[newfloat]{minted}
\captionsetup[listing]{position=top}

\begin{document}

\begin{titlepage}
	\begin{center}
    	\includegraphics[scale=0.10]{du.png}\par
		\begin{Huge}
			\textsc{University of Dhaka}\par
		\end{Huge}
		\begin{Large}
			Department of Computer Science and Engineering\par \vspace{1cm}
			CSE-3113: Microprocessor and Assembly Language Lab \\[12pt]	
			Lab Report 3\\[8pt]
		\end{Large}
	\end{center}  	
	\begin{large}
		\textbf{Submitted By:\\[12pt]}
			Name: Mahdi Mohd. Hossain Noki\\[8pt]
			Roll No : 02\\[12pt]
		\textbf{Submitted On : \\[12pt]}
			January 31, 2023\\[15pt]
		\textbf{Submitted To :\\[12pt]}
            Upama Kabir\\[10pt]
            Md. Mustafizur Rahman\\[10pt]
            Mohammad Rafiqul Islam Rafi\\[10pt]
                \newpage
	\end{large}
\end{titlepage}

\section{Task 1}
\begin{listing}[h]
Write a simple program to calculate: P = Q + R + S. Let Q = 2, R = 4, S = 5. Assume
that r1 = Q, r2 = R, r3 = S. The result P will go in r0
\begin{minted}[linenos,frame=single]{nasm}
main
	MOV r1, #2
	MOV r2, #4
	MOV r3, #5
	ADD r0,r1,r2
	ADD r0, r0, r3 
\end{minted}
The task was rather simple. Three registers - r1, r2 and r2 were loaded with constant values and ADD was used to sum up the values into r0
\end{listing}

\section{Task 2}
\begin{listing}[h]
Write a simple program to calculate: P = Q - R . Assume that r1 = Q, r2 = R, and Q¿R.
The result P will go in r0.
\begin{minted}[linenos,frame=single]{nasm}
sub
	MOV r1, #4
	MOV r2, #2
	SUB r0, r1, r2
\end{minted}
Values were loaded up on the registers r1 and r2. SUB subtracts the values and keeps the signed numbers in r0
\end{listing}

\section{Task 3}
\begin{listing}[h]
Write a simple program to calculate: P = Q - R- S. Let Q = 12, R = 4, S = 5. Assume that
r1 = Q, r2 = R, r3 = S. The result P will go in r0.
\begin{minted}[linenos,frame=single]{nasm}
moresub
	MOV r1, #12
	MOV r2, #4
	MOV r3, #5
	SUB r0, r1, r2
	SUB r0, r0, r3
\end{minted}
This was very similar to Task 2, except SUB was used twice to find r1 - r2 - r3
\end{listing}

\section{Task 4}
\begin{listing}[h]
Write a simple program to calculate: P = Q x R . The result P will go in r0.
\begin{minted}[linenos,frame=single]{nasm}
ml
	MOV r1, #12
	MOV r2, #4
	MUL r0, r1, r2
\end{minted}
This task involved multiplication. There arises a concern for overflow with sufficiently large numbers. Bigger integers needs to be handled separately to prevent overflow
\end{listing}

\section{Task 5}
\begin{listing}[h]
This problem is same as the problem 1. W = X + Y + Z . Once again, let X = 9, Y = 8,
Z = 5 and we assume that r4 = X, r3 = Y, r2 = Z. In this case, you will put the data in
memory in the form of constants before the program runs.
\begin{minted}[linenos,frame=single]{nasm}
X EQU 9
Y EQU 8
Z EQU 58
addconstants
	MOV r4, #X
	MOV r3, #Y
	MOV r2, #Z
	ADD r0, r4, r3
	ADD r0, r0, r2
\end{minted}
This task involved introducing constant values. X, Y and Z were assigned as aliases for some constants and were used later. The rest of the task is similar to Task 1.
\end{listing}
\end{document}
